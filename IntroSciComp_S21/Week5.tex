\documentclass[reqno]{amsart}


\pagestyle{empty}

\usepackage{graphicx}
\usepackage[margin = 1cm]{geometry}
\usepackage{color}
\usepackage{cancel}
\usepackage{multirow}
\usepackage{framed}
\usepackage{algorithm}
\usepackage{algorithmic}
\usepackage{amssymb}
\usepackage{stackengine}


\newtheorem{thm}{Theorem}
\newtheorem{cor}{Corollary}
\theoremstyle{definition}
\newtheorem{definition}{Definition}

\newenvironment{handwave}{%
  \renewcommand{\proofname}{Handwavey proof}\proof}{\endproof}
  %\renewcommand{\qedsymbol}{$\blacksquare$}

\begin{document}
\begin{flushleft}
{\sc \Large AMATH 301 Rahman} \hfill Week 5 Theory
\bigskip
\end{flushleft}

\newcommand{\R}{\mathbb{R}}
\newcommand{\N}{\mathbb{N}}
\newcommand{\Z}{\mathbb{Z}}
\newcommand{\Q}{\mathbb{Q}}
\renewcommand{\CancelColor}{\color{red}}
\newcommand{\?}{\stackrel{?}{=}}
\renewcommand{\varphi}{\phi}
\newcommand{\card}{\text{Card}}
\newcommand{\bigzero}{\text{\Huge 0}}
\newcommand{\curvearrowdown}{{\color{red}\rotatebox{90}{$\curvearrowleft$}}}
\newcommand{\curvearrowup}{{\color{red}\rotatebox{90}{$\curvearrowright$}}}



\section*{Week 5:  Root Finding}

For these notes you may want to read them with the lecture videos or coding lecture.  I show the illustrations in the lecture video, and do a more concrete example in the coding lecture.

This may seem silly because we have been finding roots since like 6\textsuperscript{th} grade.  However, recall that we can only do this for low order polynomials.  Further, we need to find roots of the characteristic polynomial to analyze/solve differential equations.  Plus there are other applications for this.

Let's consider a function similar to $f(x) = x^3$.

\bigskip

\underline{Section search (iterate and pray):  }

Consider the interval $[a, b]$ such that $f(a) < 0$ and $f(b) > 0$, so by the intermediate value theorem we know there is some point $x = \xi \in (a,b)$ such that $f(\xi) = 0$.  We just want to approximate the root.  One thing we can do is break up $[a,b]$ into tiny intervals, and see which $f(x_n)$ is closest to zero.  Consider
%
\begin{equation*}
[a, b] = [a, x_1]\cup [x_1, x_2] \cup \cdots \cup [x_n, x_{n+1}] \cup \cdots \cup [x_{N}, b].
\end{equation*}
%
Then test each of the nodes, $f(x_1), f(x_2), \ldots, f(x_n), \ldots, f(x_N)$.  Notice that this is quite slow.

\bigskip

\underline{Bisection:  }

Since we know the root is between $a$ and $b$, again by the intermediate value theorem we can just test the mid point: $x_1 = (a+b)/2$.  If $f(x_1) < 0$, we will redefine $a = x_1$, but if $f(x_1) > 0$, we will redefine $b = x_1$.  We repeat this until we get the desired error.  The nice thing about bisection is that it will definitely converge as long as $f(a)$ and $f(b)$ have different signs; i.e., there is a root between $a$ and $b$.

\bigskip

\underline{Newton's method:  }

Suppose we know that there is a root near $x_0$.  Lets find the tangent line of $f(x)$ at $x_0$ and record its intersection with the $x$-axis.  We repeat this process.

Newton's method also has a general formula:
%
\begin{equation*}
x_{n+1} = x_n - \frac{f(x_n)}{f'(x_n)}.
\end{equation*}
%
This is derived directly from the equation of a line.  Consider two points on a line $(x_1, y_1)$ and $(x_2, y_2)$ (for us one point will be on the curve $f(x)$ and the other point will be on the $x$-axis).  Further, recall that the slope of the tangent line is just the derivative of the function at that point.  Then
%
\begin{equation*}
m = \frac{y_1 - y_2}{x_1 - x_2} \Rightarrow f'(x_n) = \frac{f(x_n) - 0}{x_n - x_{n+1}} \Rightarrow (x_n - x_{n+1})f'(x_n) = f(x_n) \Rightarrow x_{n+1} = x_n - \frac{f(x_n)}{f'(x_n)}.
\end{equation*}

While it converges very fast, if it does converge, there are quite a few draw backs:
%
\begin{itemize}

\item  Calculating $f'(x)$ may be prohibitive,
\item  It may not converge (for example if there are many nearby roots),
\item  It converges slowly for multiplicities.

\end{itemize}



\end{document}