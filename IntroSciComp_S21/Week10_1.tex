\documentclass[reqno]{amsart}


\pagestyle{empty}

\usepackage{graphicx}
\usepackage[margin = 1cm]{geometry}
\usepackage{color}
\usepackage{cancel}
\usepackage{multirow}
\usepackage{framed}
\usepackage{algorithm}
\usepackage{algorithmic}
\usepackage{amssymb}
\usepackage{stackengine}
\usepackage{wrapfig}
\usepackage{esint}


\newtheorem{thm}{Theorem}
\newtheorem{cor}{Corollary}
\theoremstyle{definition}
\newtheorem{definition}{Definition}

\newenvironment{handwave}{%
  \renewcommand{\proofname}{Handwavey proof}\proof}{\endproof}
  %\renewcommand{\qedsymbol}{$\blacksquare$}

\begin{document}
\begin{flushleft}
{\sc \Large AMATH 301 Rahman} \hfill Week 10 Theory Part 1
\bigskip
\end{flushleft}

\newcommand{\R}{\mathbb{R}}
\newcommand{\N}{\mathbb{N}}
\newcommand{\Z}{\mathbb{Z}}
\newcommand{\Q}{\mathbb{Q}}
\renewcommand{\CancelColor}{\color{red}}
\newcommand{\?}{\stackrel{?}{=}}
\renewcommand{\varphi}{\phi}
\newcommand{\card}{\text{Card}}
\newcommand{\bigzero}{\text{\Huge 0}}
\newcommand{\curvearrowdown}{{\color{red}\rotatebox{90}{$\curvearrowleft$}}}
\newcommand{\curvearrowup}{{\color{red}\rotatebox{90}{$\curvearrowright$}}}
\renewcommand{\L}{\mathcal{L}}
\newcommand{\xvect}{\overrightarrow{x}}
\newcommand{\vect}{\overrightarrow{v}}
\newcommand{\wvect}{\overrightarrow{w}}
\newcommand{\nuvect}{\overrightarrow{\nu}}
\newcommand{\omegavect}{\overrightarrow{\omega}}
\newcommand{\del}{\nabla}


\section*{Week 10 Part 1:  Brief Introduction to Partial Differential Equations (PDEs)}

\subsection*{The Three Classical PDEs}

In the coding project I will only give a Heat equation problem, but if anyone was interested I included the other two types of PDEs.

\subsubsection*{The Heat Equation}

Consider heat conduction in some bulk space $V$ with a boundary $\partial V$.
Also consider an infinitesimal space in that bulk called $dV$.  Let $u(x,y,z,t)$
represent the temperature in $V$ at any time $t$.  Let $E = c\rho u$
where $c$ is the specific heat and $\rho$ is the mass density of the bulk,
be the total energy in $dV$.

There are some fundamental laws that will lead us to the heat equation:
%
\begin{framed}
Fourier heat conduction laws:
\begin{enumerate}
\item  If the temperature in a region is constant, there is no heat transfer in that region.
\item  Heat always flows from hot to cold.
\item  The greater the difference between temperatures at two points the faster the flow of heat
from one point to the other.
\item  The flow of heat is material dependent.
\end{enumerate}
\end{framed}
%
All these laws can be summarized into one equation
%
\begin{equation}
\phi(x,y,z,t) = -K_0 \del u(x,y,z,t)
\end{equation}
%

Now we can form a word equation:
%
\begin{equation}
\left(\text{Rate of change of heat}\right)
= \left(\text{Heat flowing into $dV$ per unit time}\right) +
\left(\text{Heat generated in $dV$ per unit time}\right)
\end{equation}
%
The first statement is the rate of change of the total energy $E$.
The second is the flux at $\partial V$ in the normal direction.  
The third is additional heat being generated in $dV$.  For the third
statement lets called the additional heat $Q$.
This gives us the equation
%
\begin{equation}
\frac{\partial}{\partial t} \iiint_V c\rho u\, dV
= - \oiint_{\partial V} \phi\cdot n\, dS
+\iiint_V Q\, dV
\end{equation}
%
And using divergence theorem we get
%
\begin{equation*}
\oiint_{\partial V} \phi\cdot n\, dS = \iiint_V \del\cdot\phi\, dV
= \iiint_V \del\cdot (-K_0\del u)\, dV = K_0\iiint_V \del^2u\, dV
\end{equation*}
%
therefore, the equation becomes
%
\begin{equation}
\frac{\partial}{\partial t} \iiint_V c\rho u\, dV
= \iiint_V c\rho \frac{\partial}{\partial t}u\, dV
= K_0\iiint_V \del^2u\, dV +\iiint_V Q\, dV
\Rightarrow c\rho \frac{\partial u}{\partial t} = K_0\del^2u + Q.
\end{equation}
%
If we consider the case $Q = 0$; i.e., no external heat being generated,
and if we divide through by $c\rho$, then we get the simplest form of the
heat equation
%
\begin{equation}
\color{red}\frac{\partial u}{\partial t} = K\del^2u
\end{equation}
%
where $K$ is called the thermal diffusivity.
%
In 1-D this is,
%
\begin{equation}
\color{red}\frac{\partial u}{\partial t} = K\frac{\partial^2 u}{\partial x^2}
\end{equation}


\subsubsection*{The Wave Equation}

Here we will only derive the 1-D version, but keeping in mind we can extend the notion
of a derivative to higher dimensions with the $\del$ operator.

Consider a vibrating string.  Let $u$ be the vertical displacement.
The slop of the string at any horizontal position $x$ is
$\tan\theta(x) = \partial u/\partial x$
where $\theta$ is the angle from the horizontal.

If we ignore any horizontal motion of the atoms we can directly use Newton's laws
for the vertical motion.  The mass of any small segment of string is $\rho\Delta x$
and the acceleration is $\partial^2 u/\partial t^2$.  The total tensile force on the
string at $x$ is $F(x,t)$ and at $\Delta x$ is $F(x+\Delta x,t)$.  We can ``pick out''
the vertical component of the force by multiplying it by sine of the angle.  Lets also
consider an external force on the string $Q$.  Then we get the equation
%
\begin{equation}
\rho\Delta x\frac{\partial^2 u}{\partial t^2} = F(x+\Delta x,t)\sin(\theta(x+\Delta x,t))
- F(x,t)\sin(\theta(x,t)) + \rho\Delta x Q(x)
\end{equation}
%
Dividing through by $\Delta x$ and taking the limit gives us
%
\begin{equation*}
\rho \frac{\partial^2 u}{\partial t^2} = \lim_{\Delta x \rightarrow 0} \frac{1}{\Delta x}
\left(F(x+\Delta x,t)\sin(\theta(x+\Delta x,t)) - F(x,t)\sin(\theta(x,t))\right)
+ \rho Q(x) = \frac{\partial}{\partial x}\left(F(x,t)\sin(\theta(x,t))\right) + \rho Q(x)
\end{equation*}
%
If our angle of deflection isn't huge, $\tan\theta = \sin\theta/\cos\theta \approx \sin\theta$,
so we can approximate sine by $\sin\theta = \partial u/\partial x$.  Let us also assume
a constant tensile force through out the string; i.e., $F = T_0$.
Then our equation simplifies to
%
\begin{equation}
\rho \frac{\partial^2 u}{\partial t^2} = \frac{\partial}{\partial x}
\left(T_0 \frac{\partial u}{\partial x}\right) + \rho Q
\Rightarrow \frac{\partial^2 u}{\partial t^2} = \frac{T_0}{\rho}
\frac{\partial^2 u}{\partial x^2} + Q
\end{equation}
%
Here $T_0/\rho = c^2$, where $c$ is the speed of propagation of the wave.  Now,
if the only external force is gravity, it is negligible compared to $T_0$, so
$Q \approx 0$.  This gives us the simplest wave equation
%
\begin{equation}
\color{red}\frac{\partial^2 u}{\partial t^2} = c^2\frac{\partial^2 u}{\partial x^2}
\end{equation}
%
This can be extended to the higher orders using the $\del$ operator
%
\begin{equation}
\color{red}\frac{\partial^2 u}{\partial t^2} = c^2\del^2 u
\end{equation}

\bigskip
\bigskip

\subsubsection*{Laplace's Equation}

We will often want to see what happens for steady state problems or when
the time derivatives of a PDE are zero.  This gives us Laplace's Equation
%
\begin{equation}
\color{red} \del^2 u = 0
\end{equation}

\bigskip
\bigskip

\subsubsection*{Boundary and Initial Condition}

Consider a boundary $x = b$, then we have some usual conditions
%
\begin{align*}
&\text{Dirichlet conditions:  } u(b,t) = B \text{  [Heat: prescribed, Wave: clamped]}\\
&\text{Neumann conditions:  }  u_x(b,t) = B \text{  [Heat: flux, Wave: sloped string]}
\end{align*}
%
We may also have a combination of these conditions.
It should be noted that for the heat equation because
we have one time derivative, it will have one initial condition.
For the wave equation since we take two time derivatives we will
have two initial conditions: one for the initial profile and the other
for the initial velocity.

\bigskip
\bigskip

\subsubsection*{Examples}

\begin{itemize}

\item[6)]  Here we need Newton's law of cooling:  The rate of change of temperature at a
point is proportional to the difference between that and the surrounding temperature; i.e.,
$\partial T/\partial t = -K(T-T_a)$, where $T_a$ is the ambient temperature.  Since the
heat transfer is not happening in the domain, but rather lateral to the domain we treat
this as external heat generation.  The problem also has insulated boundaries ($u_x = 0$
at the boundaries), and the initial temperature is a constant 100 C.  So we get
%
\begin{equation*}
\frac{\partial u}{\partial t} = D\frac{\partial^2 u}{\partial x^2} - K(u - 50);
\frac{\partial u}{\partial x}\bigg|_{x = 0} = \frac{\partial u}{\partial x}\bigg|_{x = L} = 0;
u(x,0) = 100.
\end{equation*}

\item[10)]  They say we have an external force proportional to the position, so $Q = kx$,
and the problem tells us that the string is secured at the ends ($u = 0$ at the boundaries).
For the initial condition it tells us that it is at rest ($u_t(x,0) = 0$) on the x-axis
($u(x,0) = 0$).  So we have
%
\begin{equation*}
\frac{\partial^2 u}{\partial t^2} = c^2\frac{\partial^2 u}{\partial x^2} + kx;
u(0,t) = u(L,t) = 0; u(x,0) = \frac{\partial u}{\partial t}\bigg|_{t = 0} = 0.
\end{equation*}

\item[12)]  Here the one complication is the boundary on the right.  We need a
way to express that it is 100C only after a certain value of $y$.  We do this
using the Heaviside function
%
\begin{equation*}
\frac{\partial^2 u}{\partial x^2} + \frac{\partial^2 u}{\partial y^2} = 0;
u(0,y) = e^{-y}, u(x,0) = f(x), u(\pi,y) = 100\left(1 - H_1(y)\right).
\end{equation*}

\end{itemize}

\subsection*{Heat Equation Examples}

Consider the heat equation with a generic initial condition,
%
\begin{equation}
\frac{\partial u}{\partial t} = k\frac{\partial^2 u}{\partial x^2};\quad
u(x,0) = f(x).
\end{equation}
%
with the following boundary conditions

\begin{itemize}

\item[Ex:  ]  $u(0,t) = u(L,t) = 0$.

\textbf{Solution:  }  We make the Ansatz, $u(x,t) = T(t)X(x)$.  Then we plug this
into our heat equation
%
\begin{equation*}
u_t = T'(t)X(x),\, u_{xx} = T(t)X''(x) \Rightarrow T'X = kTX''
\Rightarrow \frac{T'}{kT} = \frac{X''}{X}.
\end{equation*}
%
Since the LHS is a function of $t$ alone, and the RHS is a function of $x$ alone,
and since they are equal, they must equal a constant.  Lets call it $-\lambda^2$.
Then we have
%
\begin{equation}
\frac{T'}{kT} = \frac{X''}{X} = -\lambda^2.
\end{equation}
%
Notice that I call this from the get go because in our Sturm-Liouville problems
the negative eigenvalue case always gave us trivial solutions.  Here we bypass
that by automatically assuming a positive eigenvalue $\lambda^2$.  Now we must
solve the two differential equations.

The $T$ equation is the easiest to solve
%
\begin{equation*}
\frac{T'}{kT} = -\lambda^2 \Rightarrow T' = -k\lambda^2 T
\Rightarrow \frac{dT}{dt} =  -k\lambda^2 T \Rightarrow
\frac{dT}{T} = -k\lambda^2 dt \Rightarrow \int \frac{dT}{T} = \int -k\lambda^2 dt
\Rightarrow \ln T = -k\lambda^2 t \Rightarrow T = e^{-k\lambda^2 t}
\end{equation*}
%
Notice that we don't include the constant in front of the exponential, and that
is because the $X$ equation will have constants, and we would simply by multiplying
constants to reduce it to one constant anyway, so I choose to leave it out from the
beginning.  You don't have to though.

Now, we solve the $X$ equation by recalling our Sturm-Liouville problems
%
\begin{equation*}
\frac{X''}{X} = -\lambda^2 \Rightarrow X'' + \lambda^2 X = 0
\Rightarrow X = A\cos\lambda x + B\sin\lambda x \text{  for $\lambda \neq 0$ and  }
X = c_1x + c_2 \text{  for $\lambda = 0$.}
\end{equation*}
%
If we look at the $\lambda = 0$ case we have $X(0) = c_2 = 0$
and $X(L) = Lc_1 = 0$, so $X \equiv 0$.

Now we look at the $\lambda \neq 0$ case.  $X(0) = A = 0$ and
%
\begin{equation*}
X(L) = X(L) = B\sin\lambda x = 0 \Rightarrow \lambda = \frac{n\pi}{L}
\Rightarrow X_n = B_n\sin\frac{n\pi}{L}x \text{  and  }
T_n = e^{-k\left(\frac{n\pi}{L}\right)^2 t}
\end{equation*}

Next we combine the $T$ and $X$ solutions to get the general solutions,
%
\begin{equation}
u(x,t) = \sum_{n=1}^\infty B_n\sin\frac{n\pi x}{L}e^{-k\left(\frac{n\pi}{L}\right)^2 t}
\end{equation}
%
And we can solve for the constants using the principles from Fourier series
with the initial condition.
Since this is a Fourier sine series we have
%
\begin{equation*}
u(x,0) = \sum_{n=1}^\infty B_n\sin\frac{n\pi x}{L} = f(x)
\Rightarrow B_n = \frac{2}{L}\int_0^L f(x)\sin\frac{n\pi x}{L}dx
\end{equation*}
%
Then our full solution is
%
\begin{equation}
u(x,t) = \frac{2}{L}\sum_{n=1}^\infty \sin\frac{n\pi x}{L}e^{-k\left(\frac{n\pi}{L}\right)^2 t}
\int_0^L f(x)\sin\frac{n\pi x}{L}dx
\end{equation}


\item[Ex:  ]  $u_x(0,t) = u_x(L,t) = 0$.

\textbf{Solution:  }  We know from the first example that $T = e^{-k\lambda^2 t}$.

For the $X$ equation we need to look at our two cases.  For $\lambda = 0$ we have
$X = c_1x + c_2$, and $X'(x) = c_1$, so for both boundaries $X'(0) = c_1 = X'(L)$.
These leaves us with a constant $X = c_2$.

For the $\lambda \neq 0$ case we have
%
\begin{equation*}
X = A\cos\lambda x + B\sin\lambda x \Rightarrow
X' = -\lambda A\sin\lambda x + \lambda B\cos\lambda x
\end{equation*}
%
Then we get $X'(0) = \lambda B = 0$ and
%
\begin{equation*}
X'(L) = -\lambda A\sin\lambda L = 0 \Rightarrow \lambda = \frac{n\pi}{L}
\Rightarrow X_n = A_n\cos\frac{n\pi x}{L} \text{  and  }
T_n = e^{-k\left(\frac{n\pi}{L}\right)^2 t}
\end{equation*}
%
Next we combine the $T$ and $X$ solutions to get our general solution
%
\begin{equation}
u(x,t) = c_2 + \sum_{n=1}^\infty A_n\cos\frac{n\pi x}{L}e^{-k\left(\frac{n\pi}{L}\right)^2 t}
\end{equation}
%
Now we find our coefficients by invoking the initial condition and using
Fourier Series
%
\begin{equation*}
u(x,0) = c_2 + \sum_{n=1}^\infty A_n\cos\frac{n\pi x}{L} = f(x)
\end{equation*}
%
This gives us
%
\begin{equation*}
c_2 = \frac{1}{L}\int_0^L f(x)dx
\end{equation*}
%
and
%
\begin{equation*}
A_n = \frac{2}{L}\int_0^L f(x)\cos\frac{n\pi x}{L}dx
\end{equation*}
%
Combining everything we get the full solution
%
\begin{equation}
u(x,t) = \frac{1}{L}\int_0^L f(x)dx + \frac{2}{L}\sum_{n=1}^\infty
\cos\frac{n\pi x}{L}e^{-k\left(\frac{n\pi}{L}\right)^2 t}\int_0^L f(x)\cos\frac{n\pi x}{L}dx
\end{equation}



\item[Ex:  ]  Now lets think of heat transfer in a circle.  If we go around
in one direction we hit $x = -L$ and in the other direction $x = L$, but these
are the same point.  So we get the following boundary conditions
%
\begin{equation}
u(-L,t) = u(L,t), u_x(-L,t) = u_x(L,t)
\end{equation}

\textbf{Solution:  }  We know from the previous two problems that
our solutions will be
%
\begin{align*}
T &= e^{-k\lambda^2 t}\\
X &= c_1x + c_2 \text{  for $\lambda = 0$}\\
X &= A\cos\lambda x + B\sin\lambda x \text{  for $\lambda \neq 0$}
\end{align*}
%
For $\lambda = 0$, $X(L) = c_1L + c_2$ and $X(-L) = -c_1L + c_2$, so $c_1 = 0$.
And the derivative is trivially satisfied.

For $\lambda \neq 0$,
%
\begin{equation*}
X(L) = X(-L) \Rightarrow A\cos\lambda L + B\sin\lambda L = A\cos\lambda L - B\sin\lambda L
\Rightarrow \sin\lambda L = 0 \Rightarrow \lambda = \frac{n\pi}{L}
\end{equation*}
%
And
%
\begin{equation*}
X'(L) = X'(-L) \Rightarrow -\lambda A\sin\lambda L + \lambda B\cos\lambda L
= \lambda A\sin \lambda L + \lambda B\cos\lambda L \Rightarrow \sin\lambda L = 0
\end{equation*}
%
But we already showed this.  So, we need to keep both coefficients.
%
Then our solution for $X$, which as we saw in previous conditions (for the heat equation)
is just the initial condition of the general solution, is
%
\begin{equation}
X = c_2 + \sum_{n=1}^\infty A_n\cos\frac{n\pi x}{L} + B_n\sin\frac{n\pi x}{L} = u(x,0) = f(x)
\end{equation}
%
Now we use Fourier series to solve for the coefficients,
%
\begin{align*}
c_2 &= \frac{1}{L}\int_0^L f(x)dx\\
A_n &= \frac{2}{L}\int_0^L f(x)\cos\frac{n\pi x}{L}dx\\
B_n &= \frac{2}{L}\int_0^L f(x)\sin\frac{n\pi x}{L}dx
\end{align*}
%
Putting everything back into the general solution gives us
%
\begin{equation}
u(x,t) = \frac{1}{L}\int_0^L f(x)dx + \frac{2}{L}\sum_{n=1}^\infty
\cos\frac{n\pi x}{L}e^{-k\left(\frac{n\pi}{L}\right)^2 t}\int_0^L f(x)\cos\frac{n\pi x}{L}dx
+ \sin\frac{n\pi x}{L}e^{-k\left(\frac{n\pi}{L}\right)^2 t}\int_0^L f(x)\sin\frac{n\pi x}{L}dx
\end{equation}

\end{itemize}


\end{document}