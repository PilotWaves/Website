\documentclass[reqno]{amsart}


\pagestyle{empty}

\usepackage{graphicx}
\usepackage[margin = 1cm]{geometry}
\usepackage{color}
\usepackage{cancel}
\usepackage{multirow}
\usepackage{framed}

\newtheorem{thm}{Theorem}
\newtheorem{cor}{Corollary}
\theoremstyle{definition}
\newtheorem{definition}{Definition}

\begin{document}
\begin{flushleft}
{\sc \Large AMATH 352 Rahman} \hfill Week 2
\bigskip
\end{flushleft}

\newcommand{\R}{\mathbb{R}}
\newcommand{\N}{\mathbb{N}}
\newcommand{\Z}{\mathbb{Z}}
\newcommand{\Q}{\mathbb{Q}}
\renewcommand{\CancelColor}{\color{red}}
\newcommand{\?}{\stackrel{?}{=}}
\renewcommand{\varphi}{\phi}
\newcommand{\card}{\text{Card}}
\newcommand{\bigzero}{\text{\Huge 0}}



\section*{Sec. 1.3 Gaussian Elimination (continued)}

All of the operations we talked about up to this point can be organized into matrices.
For example, if we want to switch the first and second rows, but keep the third as is
we would do
%
\begin{equation*}
\begin{bmatrix}
0 & 1 & 0\\
1 & 0 & 0\\
0 & 0 & 1
\end{bmatrix}
\end{equation*}
%

In terms of Gaussian elimination lets go back to our original system of equations
%
\begin{equation}
Ax = \begin{pmatrix}
2 & 1 & 1\\
4 & -6 & 0\\
-2 & 7 & 2
\end{pmatrix}\begin{pmatrix}
u\\
v\\
w
\end{pmatrix} = \begin{pmatrix}
5\\
-2\\
9
\end{pmatrix} = b.
\end{equation}
%
Recall that our final system was of the form
%
\begin{equation}
Ux = \begin{pmatrix}
2 & 1 & 1\\
0 & -8 & -2\\
0 & 0 & 1
\end{pmatrix}\begin{pmatrix}
u\\
v\\
w
\end{pmatrix} = \begin{pmatrix}
5\\
-12\\
2
\end{pmatrix} = c.
\end{equation}
%
Lets see how we go from $A$ to $U$ using matrix operations.

The first operation we did was subtract 2 times the first row from the second.  Recall that we kept the
first and third rows intact, so they will be $1\quad 0 \quad 0$ and $0 \quad 0\quad 1$ respectively.
In order to do -2 times the first and add it to the second we need $-2\quad 1\quad 0$ in the second
row of the identity matrix; i.e.,
%
\begin{equation}
E = \begin{bmatrix}
1 & 0 & 0\\
-2 & 1 & 0\\
0 & 0 & 1
\end{bmatrix}
\end{equation}
In a similar fashion, the matrices for the next two operations will be
%
\begin{equation}
F = \begin{bmatrix}
1 & 0 & 0\\
0 & 1 & 0\\
1 & 0 & 1
\end{bmatrix};\qquad G = \begin{bmatrix}
1 & 0 & 0\\
0 & 1 & 0\\
0 & 1 & 1
\end{bmatrix}
\end{equation}
%
Then $GFEA = U$.  However, all we want from $U$ is to solve $x$.  Afterwards we want to get back to $A$ so
we don't lose the original matrix.  In order to write $A$ in terms of $U$ lets simply invert:
%
\begin{equation}
A = E^{-1}F^{-1}G^{-1}U.
\end{equation}
%
To invert we just do the reverse row operation for each matrix; i.e.,
%
\begin{equation}
E^{-1} = \begin{bmatrix}
1 & 0 & 0\\
2 & 1 & 0\\
0 & 0 & 1
\end{bmatrix};\qquad
F = \begin{bmatrix}
1 & 0 & 0\\
0 & 1 & 0\\
-1 & 0 & 1
\end{bmatrix};\qquad G = \begin{bmatrix}
1 & 0 & 0\\
0 & 1 & 0\\
0 & -1 & 1
\end{bmatrix}
\end{equation}
%
Multiplying the matrices gives us
%
\begin{equation}
 E^{-1}F^{-1}G^{-1} = \begin{bmatrix}
 1 & 0 & 0\\
 2 & 1 & 0\\
 -1 & -1 & 1
 \end{bmatrix} = L
\end{equation}
%
We call the matrix $L$, \underline{lower triangular}, and we call $A = LU$, \underline{LU factorization}.

Aside:  Notice that when we did Gaussian elimination, we wanted to solve $Ax = b$, but ended up
solving $Ux = c$.  This means that since $A = LU$, $LUx = b \Rightarrow Lc = b$.  This is the very
reason we are allowed to simply append the right hand side.

Now lets do a couple of examples.

\begin{enumerate}

\item[Ex:  ]  
%
\begin{equation*}
A = \begin{bmatrix}
1 & 0\\
-2 & 1
\end{bmatrix}
\end{equation*}
%
Notice that this is already in lower triangular form, so we don't have any work to do

\begin{equation*}
L = \begin{bmatrix}
1 & 0\\
-2 & 1
\end{bmatrix};  U = \begin{bmatrix}
1 & 0\\
0 & 1
\end{bmatrix}
\end{equation*}

\item[Ex:  ]  

\begin{equation*}
\begin{matrix}
\\
{\color{red}2}\\
{\color{red}-1}
\end{matrix}
\begin{bmatrix}
3 & 0 & 1\\
6 & 1 & 1\\
-3 & 1 & 0
\end{bmatrix} = \begin{matrix}
\\
\\
{\color{red}1}
\end{matrix}
\begin{bmatrix}
3 & 0 & 1\\
0 & 1 & -1\\
0 & 1 & 1
\end{bmatrix} = \begin{bmatrix}
3 & 0 & 1\\
0 & 1 & -1\\
0 & 0 & 2
\end{bmatrix} \Rightarrow U = \begin{bmatrix}
3 & 0 & 1\\
0 & 1 & -1\\
0 & 0 & 2
\end{bmatrix}
\end{equation*}
%
\begin{equation*}
E^{-1} = \begin{bmatrix}
1 & 0 & 0\\
2 & 1 & 0\\
0 & 0 & 1
\end{bmatrix},\,
F^{-1} = \begin{bmatrix}
1 & 0 & 0\\
0 & 1 & 0\\
-1 & 0 & 1
\end{bmatrix},\,
G^{-1} = \begin{bmatrix}
1 & 0 & 0\\
0 & 1 & 0\\
0 & 1 & 1
\end{bmatrix} \Rightarrow
L = \begin{bmatrix}
1 & 0 & 0\\
2 & 1 & 0\\
-1 & 1 & 1
\end{bmatrix}
\end{equation*}


\end{enumerate}

\end{document}