\documentclass[14pt]{amsart}

\pagestyle{empty}

\usepackage{graphicx}
\usepackage[margin = 1cm]{geometry}
\usepackage{color}
\usepackage{cancel}
\renewcommand{\CancelColor}{\color{red}}
\newcommand{\?}{\stackrel{?}{=}}

\newcommand*{\boxedcolor}{blue}
\makeatletter
\renewcommand{\boxed}[1]{\textcolor{\boxedcolor}{%
  \fbox{\normalcolor\m@th$\displaystyle#1$}}}
\makeatother

\newcommand{\R}{\mathbb{R}}

\begin{document}
\begin{flushleft}
{\sc \Large Math 2360 Rahman} \hfill Exam 1 Review Solutions
\bigskip
\end{flushleft}

\Large

\begin{enumerate}

\setlength\itemsep{1em}

\item  Given the system
%
\begin{equation*}
\begin{split}
3x_1 + 3x_2 + 3x_3 + 9x_4 &= b_1\\
2x_1 - x_2 + 4x_3 + 7x_4 &= b_2\\
3x_1 - 5x_2 - x_3 + 7 x_4 &= b_3
\end{split}
\end{equation*}


\begin{enumerate}

\item  Write the system in matrix form $Ax = b$.

\item  Will this have solutions for all $b \in \R^3$?  If not, which vectors $b$ give no solution?

\item  Solve the system of equations for $b = (b_1, b_2, b_3) = (3, 2, 3)$.

\textbf{Solution:  }  

\begin{align*}
\begin{matrix}
{\color{red} 1/3}\\
\\
\\
\end{matrix}
\begin{bmatrix}
3 & 3 & 3 & 9 & | & 3\\
2 & -1 & 4 & 7 & | & 2\\
3 & -5 & -1 & 7 & | & 3
\end{bmatrix} &= \begin{matrix}
\\
{\color{red} 2}\\
{\color{red} 3}
\end{matrix}\begin{bmatrix}
1 & 1 & 1 & 3 & | & 1\\
2 & -1 & 4 & 7 & | & 2\\
3 & -5 & -1 & 7 & | & 3
\end{bmatrix} = \begin{matrix}
\\
\\
{\color{red} 8/3}
\end{matrix}\begin{bmatrix}
1 & 1 & 1 & 3 & | & 1\\
0 & -3 & 2 & 1 & | & 0\\
0 & -8 & -4 & -2 & | & 0
\end{bmatrix}\\ 
&= \begin{bmatrix}
1 & 1 & 1 & 3 & | & 1\\
0 & -3 & 2 & 1 & | & 0\\
0 & 0 & -28/3 & -14/3 & | & 0
\end{bmatrix}
\end{align*}
%
Then $\boxed{x_4 = -2x_3} \Rightarrow \boxed{x_2 = 0} \Rightarrow \boxed{x_1 = 5x_3}$.


\end{enumerate}

\item  Are there any vectors $x$ such that
%
\begin{equation*}
\begin{bmatrix}
1 & 1 & 0 & -4\\
0 & 2 & 1 & 4\\
0 & 0 & 3 & 5
\end{bmatrix}x = \begin{bmatrix}
2\\
-1\\
3
\end{bmatrix}
\end{equation*}

\textbf{Solution:  }  Choose $\boxed{x_4 = 0} \Rightarrow \boxed{x_3 = 1} \Rightarrow \boxed{x_2 = -1}
\Rightarrow \boxed{x_1 = 3}$

\item  Consider the matrix
%
\begin{equation*}
A = \begin{bmatrix}
0 & 1 & -1\\
2 & -2 & -1\\
-1 & 1 & 1
\end{bmatrix}
\end{equation*}

\begin{enumerate}

\item  Find the inverse of $A$.

\begin{align*}
\begin{bmatrix}
0 & 1 & -1 & | & {\color{blue}1} & {\color{blue}0} & {\color{blue}0}\\
2 & -2 & -1 & | & {\color{blue}0} & {\color{blue}1} & {\color{blue}0}\\
-1 & 1 & 1 & | & {\color{blue}0} & {\color{blue}0} & {\color{blue}1}
\end{bmatrix} &= \begin{bmatrix}
1 & -1 & -1 & | & {\color{blue}0} & {\color{blue}0} & {\color{blue}-1}\\
0 & 1 & -1 & | & {\color{blue}1} & {\color{blue}0} & {\color{blue}0}\\
2 & -2 & -1 & | & {\color{blue}0} & {\color{blue}1} & {\color{blue}0}
\end{bmatrix} = \begin{bmatrix}
1 & -1 & -1 & | & {\color{blue}0} & {\color{blue}0} & {\color{blue}-1}\\
0 & 1 & -1 & | & {\color{blue}1} & {\color{blue}0} & {\color{blue}0}\\
0 & 0 & 1 & | & {\color{blue}0} & {\color{blue}1} & {\color{blue}2}
\end{bmatrix}\\
 &= \begin{bmatrix}
1 & -1 & 0 & | & {\color{blue}0} & {\color{blue}1} & {\color{blue}1}\\
0 & 1 & 0 & | & {\color{blue}1} & {\color{blue}1} & {\color{blue}2}\\
0 & 0 & 1 & | & {\color{blue}0} & {\color{blue}1} & {\color{blue}2}
\end{bmatrix} = \begin{bmatrix}
1 & 0 & 0 & | & {\color{blue}1} & {\color{blue}2} & {\color{blue}3}\\
0 & 1 & 0 & | & {\color{blue}1} & {\color{blue}1} & {\color{blue}2}\\
0 & 0 & 1 & | & {\color{blue}0} & {\color{blue}1} & {\color{blue}2}
\end{bmatrix} \Rightarrow \boxed{A^{-1} = \begin{bmatrix}
1 & 2 & 3\\
1 & 1 & 2\\
0 & 1 & 2
\end{bmatrix}}
\end{align*}

\item  Are the columns of $A$ linearly independent (You don't have to do any work, just explain why or why not)?

\textbf{Solution:  }  As Severus Snape would say, ``Obviously''.  We were able to take the inverse, which means the
matrix is nonsingular, and therefore the columns of $A$ must be linearly independent.

\end{enumerate}

\pagebreak

\item  Find the row echelon form of
%
\begin{equation*}
A = \begin{bmatrix}
1 & 3 & -5 & -3\\
-1 & -5 & 8 & 4\\
4 & 2 & -5 & -7\\
2 & -4 & 7 & 5
\end{bmatrix}
\end{equation*}

\textbf{Solution:  }  
%
\begin{align*}
\begin{matrix}
\\
{\color{red}-1}\\
{\color{red}4}\\
{\color{red}2}
\end{matrix}\begin{bmatrix}
1 & 3 & -5 & -3\\
-1 & -5 & 8 & 4\\
4 & 2 & -5 & -7\\
2 & -4 & 7 & 5
\end{bmatrix} = \begin{matrix}
\\
\\
{\color{red}5}\\
{\color{red}5}
\end{matrix}\begin{bmatrix}
1 & 3 & -5 & -3\\
0 & -2 & 3 & 1\\
0 & -10 & 15 & 5\\
0 & -10 & 17 & 11
\end{bmatrix} = \begin{bmatrix}
1 & 3 & -5 & -3\\
0 & -2 & 3 & 1\\
0 & 0 & 0 & 0\\
0 & 0 & 2 & 6
\end{bmatrix} = \boxed{\begin{bmatrix}
1 & 3 & -5 & -3\\
0 & -2 & 3 & 1\\
0 & 0 & 2 & 6\\
0 & 0 & 0 & 0
\end{bmatrix}}
\end{align*}

\item  Suppose we know the $LU$ factorization of some matrix $A$ to be
%
\begin{equation*}
L = \begin{bmatrix}
1 & 0 & 0 & 0\\
-2 & 1 & 0 & 0\\
1 & 2 & 1 & 0\\
-2 & -1 & 0 & 1
\end{bmatrix};\qquad U = \begin{bmatrix}
1 & 3 & -5 & -3\\
0 & -2 & 1 & 1\\
0 & 0 & 1 & 1\\
0 & 0 & 0 & 1
\end{bmatrix}
\end{equation*}
%
Solve for $Ax = b$ where
%
\begin{equation*}
b = \begin{bmatrix}
1\\
2\\
9\\
-6
\end{bmatrix}
\end{equation*}

\textbf{Solution:  }  Since we know the $LU$ factorization, we recall that $Ax = b \Leftrightarrow LUx = b
\Leftrightarrow Ux = L^{-1}b$, and
%
\begin{equation*}
L^{-1}b = \begin{bmatrix}
1 & 0 & 0 & 0\\
2 & 1 & 0 & 0\\
-5 & -2 & 1 & 0\\
4 & 1 & 0 & 1
\end{bmatrix}\begin{bmatrix}
1\\
2\\
9\\
-6
\end{bmatrix} = \begin{bmatrix}
1\\
4\\
0\\
0
\end{bmatrix}
\end{equation*}
%
Then,
%
\begin{equation*}
Ux =  \begin{bmatrix}
1 & 3 & -5 & -3\\
0 & -2 & 1 & 1\\
0 & 0 & 1 & 1\\
0 & 0 & 0 & 1
\end{bmatrix}x = \begin{bmatrix}
1\\
4\\
4\\
-2
\end{bmatrix} \Rightarrow \boxed{x_4 = 0} \Rightarrow \boxed{x_3 = 0} \Rightarrow \boxed{x_2 = -2}
\Rightarrow \boxed{x_1 = 7}.
\end{equation*}

\item  Let $A$ be and $m \times n$ matrix where $r$ is the number of its pivot columns.
What are the conditions on $m$, $n$, and $r$ (other than $r \leq m$ and $r \leq n$, which
is always true) such that $Ax = b$

\begin{enumerate}

\item  has infinitely many solutions for each $b$.

\item  has exactly one solution for each $b$.

\end{enumerate}

\pagebreak

\item  Suppose $A$ has the row echelon form $R$; i.e.,
%
\begin{equation*}
A = \begin{bmatrix}
3 & 2 & 1 & 0\\
6 & 4 & 2 & 0\\
1 & 0 & 1 & 0\\
0 & 1 & -1 & 3\\
7 & 10 & -3 & 12
\end{bmatrix};\qquad R = \begin{bmatrix}
1 & 0 & 1 & 0\\
0 & 1 & -1 & 0\\
0 & 0 & 0 & 1\\
0 & 0 & 0 & 0\\
0 & 0 & 0 & 0
\end{bmatrix}
\end{equation*}

Find the bases for the row space, column space, and null space of $A$.

\textbf{Solution:  }  Row space: $\boxed{\{(1, 0, 1, 0)$, $(0, 1, -1, 0)$, $(0, 0, 0, 1)\}}$.

Column space:  $\boxed{\{(3, 6, 1, 0, 7)$, $(2, 4, 0, 1, 10)$, $(0, 0, 0, 3, 12)\}}$.

Nullspace:  $\boxed{\{(-1, 1, 1, 0)\}}$.

\item  Let
%
\begin{equation*}
A = \begin{bmatrix}
\textbf{a}\\
\textbf{b}\\
\textbf{c}
\end{bmatrix}
\end{equation*}
%
be a $3 \times 3$ matrix with rows $\textbf{a}, \textbf{b}, \textbf{c}$, and let $\det(A) = 2$.

\begin{enumerate}

\item  Find an elementary matrix $E$ such that
%
\begin{equation*}
EA = B = \begin{bmatrix}
\textbf{c} + 3\textbf{b}\\
2\textbf{b}\\
\textbf{a}
\end{bmatrix}
\end{equation*}

\textbf{Solution:  }  Here we need to fill out the row operations in $E$,
%
\begin{equation*}
\boxed{E = \begin{bmatrix}
0 & 3 & 1\\
0 & 2 & 0\\
1 & 0 & 0
\end{bmatrix}}
\end{equation*}

\item  Compute the determinant of $B$.

\textbf{Solution:  }  $|B| = |E||A| = 2|E|$, then
%
\begin{equation*}
|B| = 
2\begin{vmatrix}
0 & 3 & 1\\
0 & 2 & 0\\
1 & 0 & 0
\end{vmatrix} = 2\cdot (-1\cdot 2\cdot 1) = \boxed{-4}.
\end{equation*}

\item  Compute the determinant of $2BA^2(B^T)^{-1}$.

\textbf{Solution:  }  $|2BA^2(B^T)^{-1}| = 2^3|B||A|^2|(B^T)^{-1}| = 2^3|B||A|^2/|B| = 2^3|A|^2 = \boxed{32}$.

\end{enumerate}

\end{enumerate}

\end{document}