\documentclass[reqno]{amsart}


\pagestyle{empty}

\usepackage{graphicx}
\usepackage[margin = 1cm]{geometry}
\usepackage{color}
\usepackage{cancel}
\usepackage{multirow}
\usepackage{framed}
\usepackage{algorithm}
\usepackage{algorithmic}
\usepackage{amssymb}
\usepackage{stackengine}


\newtheorem{thm}{Theorem}
\newtheorem{cor}{Corollary}
\theoremstyle{definition}
\newtheorem{definition}{Definition}

\newenvironment{handwave}{%
  \renewcommand{\proofname}{Handwavey proof}\proof}{\endproof}
  %\renewcommand{\qedsymbol}{$\blacksquare$}

\begin{document}
\begin{flushleft}
{\sc \Large AMATH 301 Rahman} \hfill Week 8 Theory Part 2
\bigskip
\end{flushleft}

\newcommand{\R}{\mathbb{R}}
\newcommand{\N}{\mathbb{N}}
\newcommand{\Z}{\mathbb{Z}}
\newcommand{\Q}{\mathbb{Q}}
\renewcommand{\CancelColor}{\color{red}}
\newcommand{\?}{\stackrel{?}{=}}
\renewcommand{\varphi}{\phi}
\newcommand{\card}{\text{Card}}
\newcommand{\bigzero}{\text{\Huge 0}}
\newcommand{\curvearrowdown}{{\color{red}\rotatebox{90}{$\curvearrowleft$}}}
\newcommand{\curvearrowup}{{\color{red}\rotatebox{90}{$\curvearrowright$}}}



\section*{Week 8 Part 2:  Separable ODEs}

Separable equations are the easiest equations to solve.  This why it's extremely
important to recognize separable equations.  It will save you a lot of work!
One thing we will notice right away is that Autonomous first order ODEs are
always separable.

\begin{definition}
An ODE is \underline{separable} if it can be written in the form $f(x)dx = g(y)dy$.
\end{definition}

For the next few problems we will solve some separable equations.

\begin{enumerate}

\item[Ex:  ]  $y' = x^2/y$

\textbf{Solution:  }  We separate the equation by ``moving'' $y$ to the left and $dx$ to the right,

\begin{equation*}
ydy = x^2dx \Rightarrow \frac{1}{2}y^2 = \frac{1}{3}x^3 + C_0
\Rightarrow y = \pm \sqrt{\frac{2}{3}x^3 + C_1};\; y \neq 0.
\end{equation*}

\item[Ex:  ]  $y' = \dfrac{3x^2 - 1}{3 + 2y}$

\textbf{Solution:  } We separate the equation by moving $3+2y$ to the left and
$dx$ to the right,

\begin{equation*}
(3+2y)dy = (3x^2-1)dx \Rightarrow 3y + y^2 = x^3 - x + C;\; y \neq -\frac{3}{2}.
\end{equation*}

\item[Ex:  ]  $y' = (1-2x)/y$

\textbf{Solution:  }
This type of problem is called an ``Initial Value Problem'' (IVP).  The idea
is to use the Initial Value to solve for the constant of integration.  First we separate
the problem by moving $y$ to the left and $dx$ to the right,

\begin{equation*}
ydy = (1-2x)dx \Rightarrow \frac{1}{2}y^2 = x - x^2 + C.
\end{equation*}

Now, the initial value tells us that $y = -2$ when $x = 1$, so if we plug this into
the above equation we get that $C = 2$, so plugging it back in and solving for $y$
gives,

\begin{equation*}
y = -\sqrt{2x - 2x^2 + 4};\; y \neq 0.
\end{equation*}

Notice we only chose the negative branch of the root because the initial condition
starts with negative for the $y$ value and we know that $y \neq 0$ so the solution
can't magically cross into the positive branch, so we must stay on the negative branch
for all time.

For part b and c, we did the plot in class, and the domain of existence is $-1<x<2$.

\item[Ex:  ]  $\dfrac{dy}{dt} = ty\dfrac{4-y}{1+t}$

\textbf{Solution:  }
Separating gives us,
%
\begin{align*}
&\int \frac{dy}{y(4-y)} = \int \frac{tdt}{1+t} \Rightarrow  \frac{1}{4}\int\left(\frac{1}{y} + \frac{1}{4-y}\right)dy
= \int \frac{u-1}{u}du
\Rightarrow \frac{1}{4}\left[\ln|y| - \ln|4-y|\right] = u - \ln|u| + C_0\\
&\Rightarrow \ln|\frac{y}{4-y}| = 4t - 4\ln|1+t| + C_1 \Rightarrow \frac{y}{4-y} = e^{4t}\frac{K}{(1+t)^4};
\end{align*}
%
plugging in the initial condition gives us $K = y_0/(4-y_0)$, then the full solution is
%
\begin{equation}
y = \frac{\frac{4e^{4t}y_0}{(1+t)^4(4-y_0)}}{1 + \frac{e^{4t}y_0}{(1+t)^4(4-y_0)}}
\end{equation}

\begin{enumerate}

\item  As $t \rightarrow \infty$, $y \rightarrow 4$.

\item  $y_0 = 2 \Rightarrow K = 1$, so when $y = 3.99$, $T \approx 2.84367$ (via wolfram).

\item  Now, if $t = 2$, $y/(4-y) = e^8e^{-4\ln 3}K$, then $y = 3.99 \Rightarrow y_0 \approx 3.6622$
and $y = 4.1 \Rightarrow y_0 \approx 4.4042$.  This gives us an interval of $3.6622 \leq y_0 \leq 4.4042$.

\end{enumerate}

\end{enumerate}

\end{document}