\documentclass[reqno]{amsart}


\pagestyle{empty}

\usepackage{graphicx}
\usepackage[margin = 1cm]{geometry}
\usepackage{color}
\usepackage{cancel}
\usepackage{multirow}
\usepackage{framed}
\usepackage{algorithm}
\usepackage{algorithmic}
\usepackage{amssymb}
\usepackage{stackengine}
\usepackage{wrapfig}
\usepackage{esint}


\newtheorem{thm}{Theorem}
\newtheorem{cor}{Corollary}
\theoremstyle{definition}
\newtheorem{definition}{Definition}

\newenvironment{handwave}{%
  \renewcommand{\proofname}{Handwavey proof}\proof}{\endproof}
  %\renewcommand{\qedsymbol}{$\blacksquare$}

\begin{document}
\begin{flushleft}
{\sc \Large AMATH 301 Rahman} \hfill Week 10 Theory Part 2
\bigskip
\end{flushleft}

\newcommand{\R}{\mathbb{R}}
\newcommand{\N}{\mathbb{N}}
\newcommand{\Z}{\mathbb{Z}}
\newcommand{\Q}{\mathbb{Q}}
\renewcommand{\CancelColor}{\color{red}}
\newcommand{\?}{\stackrel{?}{=}}
\renewcommand{\varphi}{\phi}
\newcommand{\card}{\text{Card}}
\newcommand{\bigzero}{\text{\Huge 0}}
\newcommand{\curvearrowdown}{{\color{red}\rotatebox{90}{$\curvearrowleft$}}}
\newcommand{\curvearrowup}{{\color{red}\rotatebox{90}{$\curvearrowright$}}}
\renewcommand{\L}{\mathcal{L}}
\newcommand{\xvect}{\overrightarrow{x}}
\newcommand{\vect}{\overrightarrow{v}}
\newcommand{\wvect}{\overrightarrow{w}}
\newcommand{\nuvect}{\overrightarrow{\nu}}
\newcommand{\omegavect}{\overrightarrow{\omega}}
\newcommand{\del}{\nabla}


\section*{Week 10 Part 2:  Heat Equation Example and Finite Differences for Heat Equation}

\subsection*{Heat Equation Examples}

Consider the heat equation with a generic initial condition,
%
\begin{equation}
\frac{\partial u}{\partial t} = k\frac{\partial^2 u}{\partial x^2};\quad
u(x,0) = f(x).
\end{equation}
%
with the following boundary conditions: $u(0,t) = u(L,t) = 0$.

\textbf{Solution:  }  We make the Ansatz, $u(x,t) = T(t)X(x)$.  Then we plug this
into our heat equation
%
\begin{equation*}
u_t = T'(t)X(x),\, u_{xx} = T(t)X''(x) \Rightarrow T'X = kTX''
\Rightarrow \frac{T'}{kT} = \frac{X''}{X}.
\end{equation*}
%
Since the LHS is a function of $t$ alone, and the RHS is a function of $x$ alone,
and since they are equal, they must equal a constant.  Lets call it $-\lambda^2$.
Then we have
%
\begin{equation}
\frac{T'}{kT} = \frac{X''}{X} = -\lambda^2.
\end{equation}
%
Notice that I call this from the get go because in our Sturm-Liouville problems
the negative eigenvalue case always gave us trivial solutions.  Here we bypass
that by automatically assuming a positive eigenvalue $\lambda^2$.  Now we must
solve the two differential equations.

The $T$ equation is the easiest to solve
%
\begin{equation*}
\frac{T'}{kT} = -\lambda^2 \Rightarrow T' = -k\lambda^2 T
\Rightarrow \frac{dT}{dt} =  -k\lambda^2 T \Rightarrow
\frac{dT}{T} = -k\lambda^2 dt \Rightarrow \int \frac{dT}{T} = \int -k\lambda^2 dt
\Rightarrow \ln T = -k\lambda^2 t \Rightarrow T = e^{-k\lambda^2 t}
\end{equation*}
%
Notice that we don't include the constant in front of the exponential, and that
is because the $X$ equation will have constants, and we would simply by multiplying
constants to reduce it to one constant anyway, so I choose to leave it out from the
beginning.  You don't have to though.

Now, we solve the $X$ equation by recalling our Sturm-Liouville problems
%
\begin{equation*}
\frac{X''}{X} = -\lambda^2 \Rightarrow X'' + \lambda^2 X = 0
\Rightarrow X = A\cos\lambda x + B\sin\lambda x \text{  for $\lambda \neq 0$ and  }
X = c_1x + c_2 \text{  for $\lambda = 0$.}
\end{equation*}
%
If we look at the $\lambda = 0$ case we have $X(0) = c_2 = 0$
and $X(L) = Lc_1 = 0$, so $X \equiv 0$.

Now we look at the $\lambda \neq 0$ case.  $X(0) = A = 0$ and
%
\begin{equation*}
X(L) = X(L) = B\sin\lambda x = 0 \Rightarrow \lambda = \frac{n\pi}{L}
\Rightarrow X_n = B_n\sin\frac{n\pi}{L}x \text{  and  }
T_n = e^{-k\left(\frac{n\pi}{L}\right)^2 t}
\end{equation*}

Next we combine the $T$ and $X$ solutions to get the general solutions,
%
\begin{equation}
u(x,t) = \sum_{n=1}^\infty B_n\sin\frac{n\pi x}{L}e^{-k\left(\frac{n\pi}{L}\right)^2 t}
\end{equation}
%
And we can solve for the constants using the principles from Fourier series
with the initial condition.
Since this is a Fourier sine series we have
%
\begin{equation*}
u(x,0) = \sum_{n=1}^\infty B_n\sin\frac{n\pi x}{L} = f(x)
\Rightarrow B_n = \frac{2}{L}\int_0^L f(x)\sin\frac{n\pi x}{L}dx
\end{equation*}
%
Then our full solution is
%
\begin{equation}
u(x,t) = \frac{2}{L}\sum_{n=1}^\infty \sin\frac{n\pi x}{L}e^{-k\left(\frac{n\pi}{L}\right)^2 t}
\int_0^L f(x)\sin\frac{n\pi x}{L}dx
\end{equation}

\pagebreak

\subsection*{Finite Differences}

The idea here is that we know how to solve boundary value problems using finite differences so we are just going to solve a boundary value problem at each timestep $n$ and iterate using a loop.

We must discretize the Heat equation 
%
\begin{equation*}
\frac{\partial u}{\partial t} = k\frac{\partial^2 u}{\partial x^2};
\end{equation*}
%
using finite differences with $i$ representing the spatial steps and $n$ representing the temporal steps.
%
\begin{equation*}
\frac{\partial u}{\partial t} = \frac{u_i^{n+1} - u_i^n}{\Delta t}
\end{equation*}
%
which is just the usual forward difference.  And
%
\begin{equation*}
k\frac{\partial^2 u}{\partial x^2} = k\left[\frac{u_{i+1}^n - 2u_i^n + u_{i-1}^n}{2\Delta x^2} + \frac{u_{i+1}^{n+1} - 2u_i^{n+1} + u_{i-1}^{n+1}}{2\Delta x^2}\right]
\end{equation*}
%
which is the average of the second derivative central difference for the next time and the previous time.  Plugging this into the PDE gives us
%
\begin{equation}
\frac{u_i^{n+1} - u_i^n}{\Delta t} = k\left[\frac{u_{i+1}^n - 2u_i^n + u_{i-1}^n}{2\Delta x^2} + \frac{u_{i+1}^{n+1} - 2u_i^{n+1} + u_{i-1}^{n+1}}{2\Delta x^2}\right]
\end{equation}
%
which can be simplified by letting $\mu = k\Delta t/(2\Delta x^2)$,
%
\begin{equation*}
u_i^{n+1} - u_i^n = \mu\left[u_{i+1}^n - 2u_i^n + u_{i-1}^n + u_{i+1}^{n+1} - 2u_i^{n+1} + u_{i-1}^{n+1}\right]
\end{equation*}
%
then we put all the time $n+1$ terms on the left hand side and all the time $n$ terms on the right hand side
%
\begin{equation*}
-\mu u_{i+1}^{n+1} + (1+2\mu)u_i^{n+1} - \mu u_{i-1}^{n+1} = \mu u_{i+1}^n + (1-2\mu)u_i^n + \mu u_{i-1}^n
\end{equation*}
%
Just like we did with ODEs, we can use tridiagonal matrices to represent this.  First lets write out each equation so we get a good idea of what needs to go into the tridiagonal entries.  It should be noted that $u_0$ and $u_M$ are just boundary terms, so lets call them $u_a$ and $u_b$ (the left and right hand boundary values).  
%
\begin{align*}
(1+2\mu)u_1^{n+1} -\mu u_2^{n+1} &= \mu u_2^n + (1-2\mu)u_1^n + 2\mu u_a\\
-\mu u_1^{n+1} + (1+2\mu)u_2^{n+1} - \mu u_3^{n+1} &= \mu u_1^n + (1-2\mu)u_2^n + \mu u_3^n\\
&\vdots\\
-\mu u_{i-1}^{n+1} + (1+2\mu)u_i^{n+1} - \mu u_{i+1}^{n+1} &= \mu u_{i-1}^n + (1-2\mu)u_i^n + \mu u_{i+1}^n\\
&\vdots\\
-\mu u_{M-3}^{n+1} + (1+2\mu)u_{M-2}^{n+1} - \mu u_{M-1}^{n+1} &= \mu u_{M-3}^n + (1-2\mu)u_{M-2}^n + \mu u_{M-1}^n\\
-\mu u_{M-2}^{n+1} + (1+2\mu)u_{M-1}^{n+1} &= \mu u_{M-2}^n + (1-2\mu)u_{M-1}^n + 2\mu u_b
\end{align*}
%
Then all we have is an $Au^{n+1} = b$ equation where $b = Bu^n + (2\mu u_a, 0, 0, \ldots, 0, 0, 2\mu u_b)$, and
%
\begin{equation}
A = \begin{pmatrix}
1 + 2\mu & -\mu & & &  \\
-\mu & 1 + 2\mu & -\mu & & \\
 & \ddots & \ddots & \ddots & \\
 & & -\mu & 1 + 2\mu & -\mu\\
 & & & -\mu & 1 + 2\mu
\end{pmatrix}; \qquad B = \begin{pmatrix}
1 - 2\mu & \mu & & &  \\
\mu & 1 - 2\mu & \mu & & \\
 & \ddots & \ddots & \ddots & \\
 & & \mu & 1 - 2\mu & \mu\\
 & & & \mu & 1 - 2\mu
\end{pmatrix}
\end{equation}
%
and therefore
%
\begin{equation}
b = Bu^n + \begin{pmatrix}
2\mu u_a\\
0\\
\vdots\\
0\\
2\mu u_b
\end{pmatrix}
\end{equation}



\end{document}