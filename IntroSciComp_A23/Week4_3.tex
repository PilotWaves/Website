\documentclass[reqno]{amsart}


\pagestyle{empty}

\usepackage{graphicx}
\usepackage[margin = 1cm]{geometry}
\usepackage{color}
\usepackage{cancel}
\usepackage{multirow}
\usepackage{framed}
\usepackage{algorithm}
\usepackage{algorithmic}
\usepackage{amssymb}
\usepackage{stackengine}


\newtheorem{thm}{Theorem}
\newtheorem{cor}{Corollary}
\theoremstyle{definition}
\newtheorem{definition}{Definition}

\newenvironment{handwave}{%
  \renewcommand{\proofname}{Handwavey proof}\proof}{\endproof}
  %\renewcommand{\qedsymbol}{$\blacksquare$}

\begin{document}
\begin{flushleft}
{\sc \Large AMATH 301 Rahman} \hfill Week 4 Theory Part 3
\bigskip
\end{flushleft}

\newcommand{\R}{\mathbb{R}}
\newcommand{\N}{\mathbb{N}}
\newcommand{\Z}{\mathbb{Z}}
\newcommand{\Q}{\mathbb{Q}}
\renewcommand{\CancelColor}{\color{red}}
\newcommand{\?}{\stackrel{?}{=}}
\renewcommand{\varphi}{\phi}
\newcommand{\card}{\text{Card}}
\newcommand{\bigzero}{\text{\Huge 0}}
\newcommand{\curvearrowdown}{{\color{red}\rotatebox{90}{$\curvearrowleft$}}}
\newcommand{\curvearrowup}{{\color{red}\rotatebox{90}{$\curvearrowright$}}}



\section*{Week 4 Part 3:  Iterative Schemes}

We talk more about iterative methods in the coding lectures, but I derive the theory here.

\underline{\color{blue}Scalar iteration}

Suppose we want to solve $(1-m)x = b; |m| < 1$, then clearly $x = b/(1-m)$.  However, we want to see how these numerical methods are derived, so lets pretend that we do not know the solution.  Notice that $(1-m)x = b$, then
%
\begin{equation}
x = mx + b
\end{equation}
%
This is just the equation of a fixed oint of a recurrence relation.  If the recurrence is stable,
%
\begin{equation}
x_{n+1} = mx_n + b
\end{equation}
%
will approximate the solution.  If we assume that $x_0 = 0$ we get
%
\begin{equation*}
x_1 = b \Rightarrow x_2 = mb + b \Rightarrow \cdots \Rightarrow x_n = m^{n-1}b + m^{n-2}b + \cdots + m^2b + mb + b.
\end{equation*}
%
This is just a geometric series in $m$, so
%
\begin{equation}
x_n = \frac{1-m^n}{1-m}b.
\end{equation}
%
Notice that if $|m| < 1$,
%
\begin{equation*}
x_n \rightarrow \frac{b}{1-m} \qquad \text{as} \qquad n \rightarrow \infty
\end{equation*}

\bigskip

\underline{\color{blue}Matrix iteration}

Similarly, for matrices, we can do $x_{n+1} = Mx_n + b$, but this gives us the fixed point equation
%
\begin{equation}
x = Mx + b \Rightarrow (I-M)x = b.
\end{equation}
%
If $A = I - M$, then this is our infamous $Ax = b$ problem.  Notice that $M^n$ has to be decreasing as $n \rightarrow \infty$ for our algorithm to converge.  Also, $I - M$ must be nonsingular.

\bigskip

\underline{\color{blue}Jacobi Method}

Let $A = L+D+U$, then
%
\begin{equation*}
Ax = (L+D+U)x = b \Rightarrow D^{-1}(L+D+U)x = D^{-1}b \Rightarrow x + D^{-1}(L+U)x = D^{-1}b,
\end{equation*}
%
which yields
%
\begin{equation}
x_{n+1} = -D^{-1}(L+U)x_n + D^{-1}b.
\end{equation}
%
Here $M = -D^{-1}(L+U)$ and $b$ is replaced with $D^{-1}b$.

\bigskip

\underline{\color{blue}Gauss-Seidel Method}

Suppose $L$ is faster to invert than $A$, and not much slower to invert than $D$, then we can invert $L+D$,
%
\begin{equation*}
(L+D)^{-1}(L+D+U)x = (L+D)^{-1}b \Rightarrow x + (L+D)^{-1}Ux = (L+D)^{-1}b,
\end{equation*}
%
and
%
\begin{equation}
x_{n+1} = -(L+D)^{-1}Ux_n + (L+D)^{-1}b.
\end{equation}

\bigskip

\underline{\color{blue}Convergence}

\begin{definition}
\color{red} An $n\times n$ matrix $A$ is said to be \underline{strictly diagonally dominant} if
%
\begin{equation}
|a_{ii}| > \sum_{j=1,\, i\neq j}^n |a_{ij}|,\quad \forall\quad 1\leq i\leq n.
\end{equation}
\end{definition}
%

While we won't prove it in this class, but it can be shown that the Jacobi and Gauss-Seidel methods are guaranteed to converge only if the matrix is strictly diagonally dominant.



\end{document}