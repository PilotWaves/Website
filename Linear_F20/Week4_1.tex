\documentclass[reqno]{amsart}


\pagestyle{empty}

\usepackage{graphicx}
\usepackage[margin = 1cm]{geometry}
\usepackage{color}
\usepackage{cancel}
\usepackage{multirow}
\usepackage{framed}
\usepackage{amssymb}
\usepackage{stackengine}

\newtheorem{thm}{Theorem}
\newtheorem{cor}{Corollary}
\theoremstyle{definition}
\newtheorem{definition}{Definition}

\begin{document}
\begin{flushleft}
{\sc \Large AMATH 352 Rahman} \hfill Week 4
\bigskip
\end{flushleft}

\newcommand{\R}{\mathbb{R}}
\newcommand{\N}{\mathbb{N}}
\newcommand{\Z}{\mathbb{Z}}
\newcommand{\Q}{\mathbb{Q}}
\renewcommand{\CancelColor}{\color{red}}
\newcommand{\?}{\stackrel{?}{=}}
\renewcommand{\varphi}{\phi}
\newcommand{\card}{\text{Card}}
\newcommand{\bigzero}{\text{\Huge 0}}
\newcommand{\curvearrowdown}{{\color{red}\rotatebox{90}{$\curvearrowleft$}}}
\newcommand{\curvearrowup}{{\color{red}\rotatebox{90}{$\curvearrowright$}}}



\section*{Sec. 1.5 Matrix Inverses}

In linear algebra we want to find a matrix $A^{-1}$ such that $AA^{-1} = A^{-1}A = I$.
We can only find an inverse if $A$ is nonsingular, so \underline{invertible} and nonsingular
mean the same thing, and \underline{noninvertible} and singular mean the same thing.
If $A$ can be put into upper triangular form, it is nonsingular, otherwise it is singular.

Note:  nonsquare matrices do not have inverses, but we can have either a right or left hand
inverse, which we will talk about later.

Now lets do an example where we find the inverse of a matrix.  Once more we use our usual $3 \times 3$
matrix
%
\begin{equation}
A = \begin{bmatrix}
2 & 1 & 1\\
4 & -6 & 0\\
-2 & 7 & 2
\end{bmatrix}
\end{equation}
%
We append the identity matrix to $A$ and use Gauss-Jordan elimination.

\begin{align*}
&\begin{matrix}
\\
{\color{red}2}\\
{\color{red}-1}
\end{matrix}\begin{bmatrix}
2 & 1 & 1 & | & {\color{blue}1} & {\color{blue}0} & {\color{blue}0}\\
4 & -6 & 0 & | & {\color{blue}0} & {\color{blue}1} & {\color{blue}0}\\
-2 & 7 & 2 & | & {\color{blue}0} & {\color{blue}0} & {\color{blue}1}
\end{bmatrix} = \begin{matrix}
\\
\\
{\color{red}-1}
\end{matrix}\begin{bmatrix}
2 & 1 & 1 & | & {\color{blue}1} & {\color{blue}0} & {\color{blue}0}\\
0 & -8 & -2 & | & {\color{blue}-2} & {\color{blue}1} & {\color{blue}0}\\
0 & 8 & 3 & | & {\color{blue}1} & {\color{blue}0} & {\color{blue}1}
\end{bmatrix} = \begin{matrix}
{\color{red}1}\\
{\color{red}-2}\\
\\
\end{matrix}\begin{bmatrix}
2 & 1 & 1 & | & {\color{blue}1} & {\color{blue}0} & {\color{blue}0}\\
0 & -8 & -2 & | & {\color{blue}-2} & {\color{blue}1} & {\color{blue}0}\\
0 & 0 & 1 & | & {\color{blue}-1} & {\color{blue}1} & {\color{blue}1}
\end{bmatrix}\\
&= \begin{matrix}
{\color{red}-1/8}\\
\\
\\
\end{matrix}\begin{bmatrix}
2 & 1 & 0 & | & {\color{blue}2} & {\color{blue}-1} & {\color{blue}-1}\\
0 & -8 & 0 & | & {\color{blue}-4} & {\color{blue}3} & {\color{blue}2}\\
0 & 0 & 1 & | & {\color{blue}-1} & {\color{blue}1} & {\color{blue}1}
\end{bmatrix} = \begin{matrix}
{\color{red}1/2}\\
{\color{red}-1/8}\\
\\
\end{matrix}\begin{bmatrix}
2 & 0 & 0 & | & {\color{blue}12/8} & {\color{blue}-5/8} & {\color{blue}-6/8}\\
0 & -8 & 0 & | & {\color{blue}-4} & {\color{blue}3} & {\color{blue}2}\\
0 & 0 & 1 & | & {\color{blue}-1} & {\color{blue}1} & {\color{blue}1}
\end{bmatrix} = \begin{bmatrix}
1 & 0 & 0 & | & {\color{blue}12/16} & {\color{blue}-5/16} & {\color{blue}-6/16}\\
0 & 1 & 0 & | & {\color{blue}4/8} & {\color{blue}-3/8} & {\color{blue}-2/8}\\
0 & 0 & 1 & | & {\color{blue}-1} & {\color{blue}1} & {\color{blue}1}
\end{bmatrix}\\
&\Rightarrow {\color{blue}A^{-1}} = \begin{bmatrix}
{\color{blue}12/16} & {\color{blue}-5/16} & {\color{blue}-6/16}\\
{\color{blue}4/8} & {\color{blue}-3/8} & {\color{blue}-2/8}\\
{\color{blue}-1} & {\color{blue}1} & {\color{blue}1}
\end{bmatrix}
\end{align*}

\begin{enumerate}

\item[Ex:  ]  
%
\begin{equation*}
A = 
\begin{bmatrix}
2 & 0\\
0 & 3
\end{bmatrix}
\end{equation*}

This one is easy since there are only diagonal terms, so
%
\begin{equation*}
A^{-1} = 
\begin{bmatrix}
1/2 & 0\\
0 & 1/3
\end{bmatrix}
\end{equation*}

\item[Ex:  ]  For this one
%
\begin{equation*}
\begin{matrix}
\\
{\color{red} 3}
\end{matrix}\begin{bmatrix}
1 & 2 & | & {\color{blue}1} & {\color{blue}0}\\
3 & 7 & | & {\color{blue}0} & {\color{blue}1}
\end{bmatrix} = \begin{matrix}
{\color{red} 2}\\
\\
\end{matrix}\begin{bmatrix}
1 & 2 & | & {\color{blue}1} & {\color{blue}0}\\
0 & 1 & | & {\color{blue}-3} & {\color{blue}1}
\end{bmatrix} = \begin{bmatrix}
1 & 0 & | & {\color{blue}7} & {\color{blue}-2}\\
0 & 1 & | & {\color{blue}-3} & {\color{blue}1}
\end{bmatrix}
\end{equation*}

Don't be fooled by the simplicity of this answer though.  Even for $2 \times 2$, the pattern may not be as you
see here.  This one is a special case since the determinant (which we will cover later) is 1.

\item[Ex:  ]  Similarly,
%
\begin{align*}
\begin{matrix}
\\
{\color{red}3}\\
{\color{red}-2}
\end{matrix}
\begin{bmatrix}
1 & 1 & 2 & | & {\color{blue}1} & {\color{blue}0} & {\color{blue}0}\\
3 & 1 & 0 & | & {\color{blue}0} & {\color{blue}1} & {\color{blue}0}\\
-2 & 0 & 3 & | & {\color{blue}0} & {\color{blue}0} & {\color{blue}1}
\end{bmatrix} = \begin{matrix}
\\
\\
{\color{red}-1}
\end{matrix}\begin{bmatrix}
1 & 1 & 2 & | & {\color{blue}1} & {\color{blue}0} & {\color{blue}0}\\
0 & -2 & -6 & | & {\color{blue}-3} & {\color{blue}1} & {\color{blue}0}\\
0 & 2 & 7 & | & {\color{blue}2} & {\color{blue}0} & {\color{blue}1}
\end{bmatrix} = \begin{matrix}
{\color{red}2}\\
{\color{red}-6}\\
\\
\end{matrix}\begin{bmatrix}
1 & 1 & 2 & | & {\color{blue}1} & {\color{blue}0} & {\color{blue}0}\\
0 & -2 & -6 & | & {\color{blue}-3} & {\color{blue}1} & {\color{blue}0}\\
0 & 0 & 1 & | & {\color{blue}-1} & {\color{blue}1} & {\color{blue}1}
\end{bmatrix}\\
= \begin{matrix}
{\color{red}-1/2}\\
\\
\\
\end{matrix}\begin{bmatrix}
1 & 1 & 0 & | & {\color{blue}3} & {\color{blue}-2} & {\color{blue}-2}\\
0 & -2 & 0 & | & {\color{blue}-9} & {\color{blue}7} & {\color{blue}6}\\
0 & 0 & 1 & | & {\color{blue}-1} & {\color{blue}1} & {\color{blue}1}
\end{bmatrix} = \begin{matrix}
\\
{\color{red}-1/2}\\
\\
\end{matrix}\begin{bmatrix}
1 & 0 & 0 & | & {\color{blue}-3/2} & {\color{blue}3/2} & {\color{blue}1}\\
0 & -2 & 0 & | & {\color{blue}-9} & {\color{blue}7} & {\color{blue}6}\\
0 & 0 & 1 & | & {\color{blue}-1} & {\color{blue}1} & {\color{blue}1}
\end{bmatrix} = \begin{bmatrix}
1 & 0 & 0 & | & {\color{blue}-3/2} & {\color{blue}3/2} & {\color{blue}1}\\
0 & 1 & 0 & | & {\color{blue}9/2} & {\color{blue}-7/2} & {\color{blue}-6/2}\\
0 & 0 & 1 & | & {\color{blue}-1} & {\color{blue}1} & {\color{blue}1}
\end{bmatrix}
\end{align*}

\end{enumerate}

Properties:
%
\begin{enumerate}
\item  $\left(A^{-1}\right)^{-1}$,
\item  $\left(A^k\right)^{-1} = \left(A^{-1}\right)^{-1}$
\item  $(cA)^{-1} = \dfrac{1}{c}A^{-1}$
\item  $\left(A^T\right)^{-1} = \left(A^{-1}\right)^T$
\item  $(AB)^{-1} = B^{-1}A^{-1}$
\end{enumerate}

\end{document}