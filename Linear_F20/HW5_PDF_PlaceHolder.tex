\documentclass[reqno]{amsart}


\pagestyle{empty}

\usepackage{graphicx}
\usepackage[margin = 1cm]{geometry}
\usepackage{color}
\usepackage{cancel}
\usepackage{multirow}
\usepackage{framed}
\usepackage{amssymb}
\usepackage{stackengine}
\usepackage{tikz}

\newtheorem{thm}{Theorem}
\newtheorem{cor}{Corollary}
\theoremstyle{definition}
\newtheorem{definition}{Definition}

\newenvironment{handwave}{%
  \renewcommand{\proofname}{Handwavey proof}\proof}{\endproof}
  %\renewcommand{\qedsymbol}{$\blacksquare$}

\begin{document}
\begin{flushleft}
{\sc \Large AMATH 352 Rahman} \hfill Homework 5 PDF Place Holder\\
Due Monday, December 7, 2020
\bigskip
\end{flushleft}

\newcommand{\R}{\mathbb{R}}
\newcommand{\N}{\mathbb{N}}
\newcommand{\Z}{\mathbb{Z}}
\newcommand{\Q}{\mathbb{Q}}
\renewcommand{\CancelColor}{\color{red}}
\newcommand{\?}{\stackrel{?}{=}}
\renewcommand{\varphi}{\phi}
\newcommand{\card}{\text{Card}}
\newcommand{\bigzero}{\text{\Huge 0}}
\newcommand{\curvearrowdown}{{\color{red}\rotatebox{90}{$\curvearrowleft$}}}
\newcommand{\curvearrowup}{{\color{red}\rotatebox{90}{$\curvearrowright$}}}

\newcommand*\circled[1]{\color{red}\tikz[baseline=(char.base)]{
            \node[shape=circle,draw,inner sep=2pt] (char) {#1};}}



\begin{enumerate}

\item  Consider the vectors
%
\begin{equation*}
u = \begin{bmatrix}
1\\
2\\
-1
\end{bmatrix};\qquad
v = \begin{bmatrix}
-1\\
2\\
-1
\end{bmatrix}
\end{equation*}

\begin{enumerate}

\item  Find the projection of $u$ onto $v$

\item  Find the projection of $v$ onto $u$

\end{enumerate}

\item  Apply the Gram-Schmidt process to transform the given basis into an orthonormal basis.
%
\begin{equation*}
\left\lbrace\begin{bmatrix}
4\\
-3
\end{bmatrix},\qquad
\begin{bmatrix}
3\\
2
\end{bmatrix}\right\rbrace
\end{equation*}

In order to make it easier to match up with the solutions that will be on the submission page on Canvas, please apply Gram-Schmidt in the order the original vectors are in.

\item  Apply the Gram-Schmidt process to transform the given basis into an orthonormal basis.
%
\begin{equation*}
\left\lbrace\begin{bmatrix}
0\\
1\\
2
\end{bmatrix},\qquad
\begin{bmatrix}
2\\
0\\
0
\end{bmatrix},\qquad
\begin{bmatrix}
1\\
1\\
1
\end{bmatrix}\right\rbrace
\end{equation*}

In order to make it easier to match up with the solutions that will be on the submission page on Canvas, please apply Gram-Schmidt in the order the original vectors are in.

\item  Find the least squares solution, $\hat{x}$ to
%
\begin{equation*}
\begin{bmatrix}
1 & -1 & 1\\
1 & 1 & 1\\
0 & 1 & 1\\
1 & 0 & 1
\end{bmatrix}x = \begin{bmatrix}
2\\
1\\
0\\
2
\end{bmatrix}
\end{equation*}

\item  Find the least squares solution, $\hat{x}$ to
%
\begin{equation*}
\begin{bmatrix}
0 & 2 & 1\\
1 & 1 & -1\\
2 & 1 & 0\\
1 & 1 & 1\\
0 & 2 & -1
\end{bmatrix}x = \begin{bmatrix}
1\\
0\\
1\\
-1\\
0
\end{bmatrix}
\end{equation*}

\item  Find a quadratic fit for data points $(-2, 0),\, (-1,0),\, (0,1),\, (1,2),\, (2,5)$.
This is similar to finding a linear fit, which I showed in the lectures.  For a linear fit
we use the equation $y = mx + b$ to write down our matrix equation.  For a quadratic fit
we will use the same idea, except with the equation $y = ax^2 + bx + c$.  Notice that the
matrix for the quadratic fit will have three columns.  

I didn't show this in the lecture or notes because I want you to practice going from something
we did (linear fit) to something we didn't (quadratic fit) using the ideas of the linear fit.
I will put a problem on the Final that goes a few steps beyond, but you can again use the same
ideas.

\item  Solve the eigenvalue problem for the matrix.  Please find the eigenvalues and eigenvectors.
%
\begin{equation*}
\begin{bmatrix}
2 & -2 & 3\\
0 & 3 & -2\\
0 & -1 & 2
\end{bmatrix}
\end{equation*}

\item  Solve the eigenvalue problem for the matrix.  Please find the eigenvalues and eigenvectors.
%
\begin{equation*}
\begin{bmatrix}
0 & -3 & 5\\
-4 & 4 & -10\\
0 & 0 & 4
\end{bmatrix}
\end{equation*}

Please refrain from exchanging rows as that will give you the wrong answer because with
any row exchange you would be solving the problem $PAx = \lambda x$, which is not equivalent
to the problem $Ax = \lambda x$.

\item  If a diagonalization exists, find the matrices $S$ and $\Lambda$ from $A = S\Lambda S^{-1}$.
If a diagonalization does not exist, state the reasoning.
%
\begin{equation*}
A = \begin{bmatrix}
2 & -2 & 3\\
0 & 3 & -2\\
0 & -1 & 2
\end{bmatrix}
\end{equation*}

\item  If a diagonalization exists, find the matrices $S$ and $\Lambda$ from $A = S\Lambda S^{-1}$.
If a diagonalization does not exist, state the reasoning.
%
\begin{equation*}
A = \begin{bmatrix}
1 & -2 & 1\\
0 & 1 & 4\\
0 & 0 & 2
\end{bmatrix}
\end{equation*}

\end{enumerate}


\end{document}