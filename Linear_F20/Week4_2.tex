\documentclass[reqno]{amsart}


\pagestyle{empty}

\usepackage{graphicx}
\usepackage[margin = 1cm]{geometry}
\usepackage{color}
\usepackage{cancel}
\usepackage{multirow}
\usepackage{framed}
\usepackage{amssymb}
\usepackage{stackengine}

\newtheorem{thm}{Theorem}
\newtheorem{cor}{Corollary}
\theoremstyle{definition}
\newtheorem{definition}{Definition}

\begin{document}
\begin{flushleft}
{\sc \Large AMATH 352 Rahman} \hfill Week 4
\bigskip
\end{flushleft}

\newcommand{\R}{\mathbb{R}}
\newcommand{\N}{\mathbb{N}}
\newcommand{\Z}{\mathbb{Z}}
\newcommand{\Q}{\mathbb{Q}}
\renewcommand{\CancelColor}{\color{red}}
\newcommand{\?}{\stackrel{?}{=}}
\renewcommand{\varphi}{\phi}
\newcommand{\card}{\text{Card}}
\newcommand{\bigzero}{\text{\Huge 0}}
\newcommand{\curvearrowdown}{{\color{red}\rotatebox{90}{$\curvearrowleft$}}}
\newcommand{\curvearrowup}{{\color{red}\rotatebox{90}{$\curvearrowright$}}}



\section*{Sec. 1.6 Transposes and Symmetric Matrices}

{\color{red}If $A = A^T$ then it is symmetric.}

If $A = LU$ (i.e., $A$ is regular), and we multiply the rows of $U$ by a scalar factor, the reciprocals of the scalar factor can be saved in a matrix $D$.  That is,
%
\begin{equation*}
U = \begin{bmatrix}
a & b\\
0 & c
\end{bmatrix} \Rightarrow DV = \begin{bmatrix}
a & 0\\
0 & c
\end{bmatrix}\begin{bmatrix}
1 & b/a\\
0 & 1
\end{bmatrix}
\end{equation*}
%
If we further assume that $A$ is symmetric, then $V = L^T$ and $A = LDL^T$.  {\color{red}We'll see more neat properties when we do Gram-Schmidt}.

Also note that if $P$ is a permutation matrix, then $PP^T = I$.

\begin{equation*}
P = \begin{bmatrix}
0 & 1 & 0\\
0 & 0 & 1\\
1 & 0 & 0
\end{bmatrix} \Rightarrow P^T = \begin{bmatrix}
0 & 0 & 1\\
1 & 0 & 0\\
0 & 1 & 0
\end{bmatrix} \Rightarrow PP^T
\end{equation*}

\end{document}