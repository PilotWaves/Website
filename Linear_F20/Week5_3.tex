\documentclass[reqno]{amsart}


\pagestyle{empty}

\usepackage{graphicx}
\usepackage[margin = 1cm]{geometry}
\usepackage{color}
\usepackage{cancel}
\usepackage{multirow}
\usepackage{framed}
\usepackage{amssymb}
\usepackage{stackengine}

\newtheorem{thm}{Theorem}
\newtheorem{cor}{Corollary}
\theoremstyle{definition}
\newtheorem{definition}{Definition}

\newenvironment{handwave}{%
  \renewcommand{\proofname}{Handwavey proof}\proof}{\endproof}
  %\renewcommand{\qedsymbol}{$\blacksquare$}

\begin{document}
\begin{flushleft}
{\sc \Large AMATH 352 Rahman} \hfill Week 5
\bigskip
\end{flushleft}

\newcommand{\R}{\mathbb{R}}
\newcommand{\N}{\mathbb{N}}
\newcommand{\Z}{\mathbb{Z}}
\newcommand{\Q}{\mathbb{Q}}
\renewcommand{\CancelColor}{\color{red}}
\newcommand{\?}{\stackrel{?}{=}}
\renewcommand{\varphi}{\phi}
\newcommand{\card}{\text{Card}}
\newcommand{\bigzero}{\text{\Huge 0}}
\newcommand{\curvearrowdown}{{\color{red}\rotatebox{90}{$\curvearrowleft$}}}
\newcommand{\curvearrowup}{{\color{red}\rotatebox{90}{$\curvearrowright$}}}



\section*{Sec. 2.1 and 2.2 Vector spaces and subspaces}

Here we saw some definitions and how they apply to some examples.

\begin{definition}
The vectors $v_1,\ldots,v_n$ are said to be \underline{Linearly Independent} if
$c_1v_1 + \cdots + c_nv_n \neq 0$ whenever $c_i \neq 0$ for $i = 1,\ldots,n$; otherwise
they are said to be \underline{Linearly Dependent}.
\end{definition}

\begin{definition}
The expression $c_1v_1 + \cdots + c_nv_n$ is said to be a \underline{Linear Combination} of $v_1,\ldots,v_n$.
\end{definition}

A vector space is simply a space that contains all of the axioms of vector addition and scalar multiplication, and
is self-contained; i.e., addition and scalar multiplication of any combination of vectors will produce a vector
in that space.

\begin{definition}
A \underline{subspace} of a vector space is a nonempty subset that satisfies the requirements for a vector space:
Linear combinations stay in the subset;
%
\begin{itemize}

\item[(i)  ]  if we add any vectors $x$ and $y$ in the subspace $x+y$ is in the subspace,

\item[(ii)]  if we multiply any vector $x$ in the subspace by any scalar $c$, $cx$ is in the subspace.

\end{itemize}
\label{Def: Subspace}
\end{definition}

\bigskip
\bigskip
\bigskip
\bigskip

Now lets look at some examples

\begin{enumerate}

\setlength{\itemsep}{2em}

\item[Ex:  ]  $W \subseteq \R^4$ such that $\forall\quad x,y,z \in \R$,
%
\begin{equation*}
\begin{vmatrix}
x\\
y\\
z\\
0
\end{vmatrix} \in W.
\end{equation*}

$W$ is clearly nonempty and a subset of $V$.  We just have to check the properties listed in
Def. \ref{Def: Subspace}.

\begin{itemize}

\item[(i)  ]

\begin{equation*}
\begin{bmatrix}
x_1\\
x_2\\
x_3\\
0
\end{bmatrix} + \begin{bmatrix}
y_1\\
y_2\\
y_3\\
0
\end{bmatrix} = \begin{bmatrix}
x_1 + y_1\\
x_2 + y_2\\
x_3 + y_3\\
0
\end{bmatrix}
\end{equation*}
%
By the axioms of arithmetic $x_i + y_i$ will be real numbers, and the last entry is zero, so this is in $W$.

\item[(ii)  ]  

\begin{equation*}
c\begin{bmatrix}
x_1\\
x_2\\
x_3\\
0
\end{bmatrix} = \begin{bmatrix}
cx_1\\
cx_2\\
cx_3\\
0
\end{bmatrix}
\end{equation*}
%
Again, by the axioms of arithmetic $cx_i$ will be real numbers, and the last entry is zero, so this is in $W$ as well.

\end{itemize}

Since both properties are satisfied, $W$ is a subspace of $\R^4$.


\item[Ex:  ]  $V \subseteq \R^3$ such that $\forall\quad x,y\in \R$,
%
\begin{equation*}
\begin{vmatrix}
x\\
y\\
4x-5y
\end{vmatrix}\in V.
\end{equation*}

Just as the previous problem

\begin{itemize}

\item[(i)  ]  

\begin{equation*}
\begin{bmatrix}
x_1\\
x_2\\
4x_1 - 5y_1
\end{bmatrix} + \begin{bmatrix}
y_1\\
y_2\\
y_3
\end{bmatrix} = \begin{bmatrix}
x_1 + y_1\\
x_2 + y_2\\
4(x_1 + x_2) - 5(y_1 + y_2)
\end{bmatrix}
\end{equation*}
%
By the axioms of arithmetic all three entries will be real, thus matching the definition of the set $V$, and hence the vector is in $V$.

\item[(ii)  ]  

\begin{equation*}
c\begin{bmatrix}
x_1\\
x_2\\
4x_1 - 5y_1
\end{bmatrix} = \begin{bmatrix}
cx_1\\
cx_2\\
4cx_1 - 5cy_1
\end{bmatrix} \in V
\end{equation*}
%
So, this too satisfies our conditions.

Therefore, $V$ is a subspace in $\R^3$.

\end{itemize}

\item[Ex:  ]  $W \subseteq \R^3$ such that $\forall\quad x,y\in \R$,
%
\begin{equation*}
\begin{vmatrix}
x\\
y\\
-1
\end{vmatrix} \in W
\end{equation*}

Here both properties can be violated.  For property (ii),
%
\begin{equation*}
c\begin{bmatrix}
x\\
y\\
-1
\end{bmatrix} = \begin{bmatrix}
cx\\
cy\\
-c
\end{bmatrix}
\end{equation*}
%
In general $-c \neq -1$, so this vector cannot be in $W$.  Hence, $W$ is not a subspace of $\R^3$.

\item[Ex:  ]  $V \subseteq \R^2$ such that $\forall\quad x,y\in \R$,
%
\begin{equation*}
\begin{vmatrix}
x\\
y
\end{vmatrix} \in V.
\end{equation*}


Here only property (ii) is violated,
%
\begin{equation*}
\sqrt{2}\begin{bmatrix}
1\\
1
\end{bmatrix} = \begin{bmatrix}
\sqrt{2}\\
\sqrt{2}
\end{bmatrix} \notin Q,
\end{equation*}
%
therefore is not in $V$, and $V$ is not a subspace of $\R^2$.

\end{enumerate}

\end{document}