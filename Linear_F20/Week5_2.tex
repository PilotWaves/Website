\documentclass[reqno]{amsart}


\pagestyle{empty}

\usepackage{graphicx}
\usepackage[margin = 1cm]{geometry}
\usepackage{color}
\usepackage{cancel}
\usepackage{multirow}
\usepackage{framed}
\usepackage{amssymb}
\usepackage{stackengine}

\newtheorem{thm}{Theorem}
\newtheorem{cor}{Corollary}
\theoremstyle{definition}
\newtheorem{definition}{Definition}

\newenvironment{handwave}{%
  \renewcommand{\proofname}{Handwavey proof}\proof}{\endproof}
  %\renewcommand{\qedsymbol}{$\blacksquare$}

\begin{document}
\begin{flushleft}
{\sc \Large AMATH 352 Rahman} \hfill Week 5
\bigskip
\end{flushleft}

\newcommand{\R}{\mathbb{R}}
\newcommand{\N}{\mathbb{N}}
\newcommand{\Z}{\mathbb{Z}}
\newcommand{\Q}{\mathbb{Q}}
\renewcommand{\CancelColor}{\color{red}}
\newcommand{\?}{\stackrel{?}{=}}
\renewcommand{\varphi}{\phi}
\newcommand{\card}{\text{Card}}
\newcommand{\bigzero}{\text{\Huge 0}}
\newcommand{\curvearrowdown}{{\color{red}\rotatebox{90}{$\curvearrowleft$}}}
\newcommand{\curvearrowup}{{\color{red}\rotatebox{90}{$\curvearrowright$}}}



\section*{Sec. 1.9 Determinants}

The determinant is a way to measure the ``size'' of a matrix.  Think of it as the analog of absolute value in real
numbers, or the modulus in complex numbers.  Lets look at examples of $2 \times 2$ and $3 \times 3$ matrices.
These can be easily extended to $n \times n$.

\underline{$2 \times 2$}

\begin{equation}
\det \begin{pmatrix}
a_{11} & a_{12}\\
a_{21} & a_{22}
\end{pmatrix} = \begin{vmatrix}
a_{11} & a_{12}\\
a_{21} & a_{22}
\end{vmatrix} = a_{11}a_{22} - a_{12}a_{21}
\end{equation}

\underline{$3 \times 3$}

For higher order determinants such as $3 \times 3$, we use cofactor expansion.  While there are some
``shortcuts'', I personally don't see them saving much time.

\begin{equation}
\det \begin{pmatrix}
a_{11} & a_{12} & a_{13}\\
a_{21} & a_{22} & a_{23}\\
a_{13} & a_{32} & a_{33}
\end{pmatrix} = \begin{vmatrix}
a_{11} & a_{12} & a_{13}\\
a_{21} & a_{22} & a_{23}\\
a_{13} & a_{32} & a_{33}
\end{vmatrix} = {\color{red}a_{11}}{\color{blue}\begin{vmatrix}
a_{22} & a_{23}\\
a_{32} & a_{33}
\end{vmatrix}}
-  {\color{red}a_{21}}{\color{blue}\begin{vmatrix}
a_{21} & a_{23}\\
a_{31} & a_{33}
\end{vmatrix}} + {\color{red}a_{13}}{\color{blue}\begin{vmatrix}
a_{21} & a_{22}\\
a_{31} & a_{32}
\end{vmatrix}}
\end{equation}
%
Then just take the determinants of each $2 \times 2$ matrix and you are done!
For a $4 \times 4$ matrix you do the same thing, but you would get $3 \times 3$ cofactors.
We did a generic example in the lecture.

Now lets look at some examples

\begin{enumerate}

\item[Ex:  ] 

\begin{equation*}
\begin{vmatrix}
2 & 1\\
3 & 4
\end{vmatrix} = 8 - 3 = 5.
\end{equation*}

\item[Ex:  ]

\begin{equation*}
\begin{vmatrix}
5 & 2\\
-6 & 3
\end{vmatrix} = 15 + 12 = 27
\end{equation*}

\item[Ex:  ]  

\begin{equation*}
\begin{vmatrix}
1 & 4 & -2\\
3 & 2 & 0\\
-1 & 4 & 3
\end{vmatrix} = {\color{red}1}{\color{blue}\begin{vmatrix}
2 & 0\\
4 & 3
\end{vmatrix}}
-  {\color{red}4}{\color{blue}\begin{vmatrix}
3 & 0\\
-1 & 3
\end{vmatrix}} + {\color{red}(-2)}{\color{blue}\begin{vmatrix}
3 & 2\\
-1 & 4
\end{vmatrix}} = -58
\end{equation*}

\item[Ex:  ]  

\begin{equation*}
\begin{vmatrix}
2 & 4 & 6\\
0 & 3 & 1\\
0 & 0 & -5
\end{vmatrix} = 2\begin{vmatrix}
3 & 1\\
0 & -5
\end{vmatrix} = 2\cdot 3\cdot (-5).
\end{equation*}

Notice that since we have an upper triangular matrix, all we need to do is take the product of the diagonal: $\det = 2\cdot 3\cdot (-5) = -30$

\end{enumerate}

\underline{Properties of Determinants}

We will list these generically, but illustrate with $2 \times 2$ matrices.

\begin{enumerate}

\item  $\det(I) = 1$.

E.g.,
%
\begin{equation*}
\begin{vmatrix}
1 & 0\\
0 &1
\end{vmatrix} = 1; \qquad \begin{vmatrix}
1 & 0 & 0\\
0 & 1 & 0\\
0 & 0 & 1
\end{vmatrix} = 1; \qquad \text{etc.}
\end{equation*}

\item  Row exchange changes sign of the determinant.

E.g.,
\begin{equation*}
\begin{vmatrix}
a & b\\
c & d
\end{vmatrix} = ad - bc = -(bc - ad) = -\begin{vmatrix}
c & d\\
a & b
\end{vmatrix}
\end{equation*}

\item  If two rows of $A$ are equal, then $\det(A) = 0$

E.g.,
%
\begin{equation*}
\begin{vmatrix}
a & b\\
a & b
\end{vmatrix} = ab - ba = 0.
\end{equation*}

\item  If $A$ has a row (or column) of zeros, then $\det(A) = 0$.

\begin{equation*}
\begin{vmatrix}
0 & 0\\
c & d
\end{vmatrix} = 0;\qquad \begin{vmatrix}
0 & b\\
0 & d
\end{vmatrix} = 0.
\end{equation*}

\pagebreak

\item  If $A_{n \times n}$ is triangular (either lower or upper) then $\det(A) = a_{11}a_{22}a_{33}\cdots a_{nn}$; i.e.,
take the product of the diagonal elements.

E.g.,
\begin{equation*}
\begin{vmatrix}
a & 0\\
c & d
\end{vmatrix} = ad;\qquad \begin{vmatrix}
a & b\\
0 & d
\end{vmatrix} = ad.
\end{equation*}

Can we prove this for $n \times n$ matrices?

\begin{handwave}
Consider,
%
\begin{equation*}
D = \begin{bmatrix}
a_{11} & \cdots & \cdots & \cdots & \cdots \\
 & a_{22} & \cdots & \cdots & \cdots \\
 & & \ddots & & \\
 &\bigzero & & \ddots &  \\
 & & & & a_{nn}
\end{bmatrix}
\end{equation*}
%
Then
%
\begin{align*}
D &= \begin{vmatrix}
a_{11} & \cdots & \cdots & \cdots & \cdots \\
 & a_{22} & \cdots & \cdots & \cdots \\
 & & \ddots & & \\
 &\bigzero & & \ddots &  \\
 & & & & a_{nn}
\end{vmatrix} = {\color{red}a_{11}}\begin{vmatrix}
a_{22} & \cdots & \cdots & \cdots & \cdots \\
 & a_{33} & \cdots & \cdots & \cdots \\
 & & \ddots & & \\
 &\bigzero & & \ddots &  \\
 & & & & a_{nn}
\end{vmatrix} = {\color{red}a_{11}}{\color{red}a_{22}}\begin{vmatrix}
a_{33} & \cdots & \cdots & \cdots & \cdots \\
 & a_{44} & \cdots & \cdots & \cdots \\
 & & \ddots & & \\
 &\bigzero & & \ddots &  \\
 & & & & a_{nn}
\end{vmatrix}\\
&= {\color{red}a_{11}}{\color{red}a_{22}}{\color{red}a_{33}}\begin{vmatrix}
a_{44} & \cdots & \cdots & \cdots & \cdots \\
 & a_{55} & \cdots & \cdots & \cdots \\
 & & \ddots & & \\
 &\bigzero & & \ddots &  \\
 & & & & a_{nn}
 \end{vmatrix}
 = \cdots = {\color{red}a_{11}}{\color{red}a_{22}}{\color{red}a_{33}}\cdots{\color{red}a_{(n-1)(n-1)}}{\color{red}a_{nn}}.
\end{align*}

\end{handwave}

\item  If $A$ is singular, then $\det(A) = 0$; otherwise $\det(A) \neq 0$, then $A$ is invertible.
If $A$ is in row-echelon form (not to be confused with the full augmented matrix), but not upper triangular,
then at least one of the pivots will be zero.

\item  Product rule: $|A||B| = |AB|$.

\begin{equation*}
\begin{vmatrix}
a & b\\
c & d
\end{vmatrix}\begin{vmatrix}
e & f\\
g & h
\end{vmatrix} = (ad - bc)(eh - fg) = (ae+bg)(cf+dh) - ( af+bh)(ce+dg) = \begin{vmatrix}
ae+bg & af+bh\\
ce+dg & cf+dh
\end{vmatrix}
\end{equation*}

However, for scalar multiplication, $|cD_{n \times n}| = c^n|D_{n \times n}|$ because this can be written as
%
\begin{equation*}
cD_{n \times n} = \begin{bmatrix}
c &  &  & \\
 & c & \bigzero & \\
\bigzero &  & \ddots & \\
 & & & c
\end{bmatrix}
\end{equation*}

A special case is $|A||A^{-1}| = |AA^{-1}| = |I| = 1 \Rightarrow |A^{-1}| = 1/|A|$.

\item  Transpose:  $|A| = |A^T|$.

E.g.,
%
\begin{equation*}
|A| = \begin{vmatrix}
a & b\\
c & d
\end{vmatrix} = \begin{vmatrix}
a & c\\
b & d
\end{vmatrix} = |A^T|.
\end{equation*}


\end{enumerate}




\end{document}